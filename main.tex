\documentclass[11pt,t]{beamer}
\usepackage{graphicx}
\setbeameroption{hide notes}
\setbeamertemplate{note page}[plain]

\usetheme{default}
\beamertemplatenavigationsymbolsempty
\hypersetup{pdfpagemode=UseNone} % don't show bookmarks on initial view

% font
\usepackage{fontspec}
\setsansfont{TeX Gyre Heros}
\setbeamerfont{note page}{family*=pplx,size=\footnotesize} % Palatino for notes

\definecolor{offwhite}{RGB}{249,242,215}
\definecolor{foreground}{RGB}{255,255,255}
\definecolor{background}{RGB}{24,24,24}
\definecolor{title}{RGB}{107,174,214}
\definecolor{gray}{RGB}{155,155,155}
\definecolor{subtitle}{RGB}{102,255,204}
\definecolor{hilight}{RGB}{102,255,204}
\definecolor{vhilight}{RGB}{255,111,207}
\definecolor{lolight}{RGB}{155,155,155}

% use those colors
\setbeamercolor{titlelike}{fg=title}
\setbeamercolor{subtitle}{fg=subtitle}
\setbeamercolor{institute}{fg=gray}
\setbeamercolor{normal text}{fg=foreground,bg=background}
\setbeamercolor{item}{fg=foreground} % color of bullets
\setbeamercolor{subitem}{fg=gray}
\setbeamercolor{itemize/enumerate subbody}{fg=gray}
\setbeamertemplate{itemize subitem}{{\textendash}}
\setbeamerfont{itemize/enumerate subbody}{size=\footnotesize}
\setbeamerfont{itemize/enumerate subitem}{size=\footnotesize}

% page number
\setbeamertemplate{footline}{%
    \raisebox{5pt}{\makebox[\paperwidth]{\hfill\makebox[20pt]{\color{gray}
          \scriptsize\insertframenumber}}}\hspace*{5pt}}

% add a bit of space at the top of the notes page
\addtobeamertemplate{note page}{\setlength{\parskip}{12pt}}

% title info
\title{Avances de la tesina}
%\subtitle{A researcher's perspective}
\author{Isaí E. Dávila Cuba}
%\institute{\href{https://www.biostat.wisc.edu}{Biostatistics \& Medical Informatics} \\[2pt] \href{http://www.wisc.edu}{University of Wisconsin{\textendash}Madison}}
\date{28 de junio del 2021}

\input{shortcuts}



\usepackage[english]{babel}
\usepackage[utf8x]{inputenc}


\usepackage{amsthm, amssymb, amsfonts, amsmath}
\usepackage{graphicx}
\usepackage{tikz}
\usetikzlibrary{calc,shapes}
% \usepackage{enumitem}
\usepackage{mathtools}
\usepackage{mathrsfs}
\usepackage{tikz-cd}


\begin{document}

\begin{frame}
  \titlepage
\end{frame}

\begin{frame}{Modelo abeliano de Higgs}

$\phi:\m R^{2+1}\ra\m C$ es un campo escalar complejo acoplado al campo electromagnético mediante acoplamiento mínimo
\begin{align}
    \mc L = -\f 14 F_{\mu\nu}F^{\mu\nu}+\f 12D_\mu\phi\overline{D^\mu\phi}-\f{\lambda}8(1-|\phi|^2)^2
\end{align}
Ecuaciones de Euler-Lagrange
\begin{align}
    D_\mu D^\mu\phi+\f{\lambda}2(1-|\phi|^2)\phi &= 0\\
    \pr_\mu F^{\mu\nu} &= J^\nu
\end{align}
donde $J^\mu=\f{i}2(\bar\phi D^\mu\phi-\phi\overline{D^\mu\phi})$.
Buscamos soluciones estáticas y con energía finita. Definimos el funcional de energía $V_\lambda$
\begin{align}
    V_\lambda = \f 12\int_{\m R^2}\bp{B^2+D_i\phi\overline{D_i\phi}+\f\lambda 4(1-|\phi|^2)^2}d^2x
\end{align}

\end{frame}

\begin{frame}

Condiciones de contorno en el infinito ($|\vct x|\ra\infty$) para que $V_\lambda$ sea finita
\begin{align}
    |\phi|&\ra 1\\
    B&\ra 0\\
    D_i\phi &\ra 0
\end{align}
Estas condiciones de contorno permiten escribir el \emph{ansatz} de Nielsen-Olesen
\begin{align}
    \phi(\rho,\theta) &= f_N(\rho)e^{iN\theta}\\
    A_\theta(\rho,\theta) &= N\al_N(\rho)\\
    A_\rho &= 0
\end{align}
\begin{align}
    \f{d^2f_N}{d\rho^2}+\f{1}{\rho}\f{df_N}{d\rho}-\f{1}{\rho^2}(N-\al_N)^2 f_N+\f{\lambda}2(1-f_N^2)f_N &=0\\
    \f{d^2\al_N}{d\rho^2}-\f{1}\rho\f{d\al_N}{d\rho}+(N-\al_N)f_N^2 &= 0
\end{align}
    
\end{frame}

\begin{frame}{Soluciones numéricas}

\includegraphics[width=0.5\textwidth]{fields.png}
\includegraphics[width=0.5\textwidth]{Bfield.png}
\includegraphics[width=0.5\textwidth]{Energy.png}
    
\end{frame}



% \begin{frame}
% % \frametitle{d}
%    \onslide<1->{
%    } % onslide 2
%    \onslide<1->{
%    } % onslide 3
%    \onslide<1->{
%    } % onslide 4
%    \onslide<1->{
%    } % onslide 5
%    \onslide<1->{
%    } % onslide 6
%    \onslide<1->{
%    } % onslide 7
%    \onslide<1->{
%    } % onslide 8
%    \onslide<1->{
%    } % onslide 9
%    \onslide<1->{
%    } % onslide 10
% \end{frame}


% \begin{frame}
% % \frametitle{d}
%    \onslide<1->{
%    } % onslide 2
%    \onslide<1->{
%    } % onslide 3
%    \onslide<1->{
%    } % onslide 4
%    \onslide<1->{
%    } % onslide 5
%    \onslide<1->{
%    } % onslide 6
%    \onslide<1->{
%    } % onslide 7
%    \onslide<1->{
%    } % onslide 8
%    \onslide<1->{
%    } % onslide 9
%    \onslide<1->{
%    } % onslide 10
% \end{frame}



% \begin{frame}
% % \frametitle{d}
%    \onslide<1->{
%    } % onslide 2
%    \onslide<1->{
%    } % onslide 3
%    \onslide<1->{
%    } % onslide 4
%    \onslide<1->{
%    } % onslide 5
%    \onslide<1->{
%    } % onslide 6
%    \onslide<1->{
%    } % onslide 7
%    \onslide<1->{
%    } % onslide 8
%    \onslide<1->{
%    } % onslide 9
%    \onslide<1->{
%    } % onslide 10
% \end{frame}



% \begin{frame}
% % \frametitle{d}
%    \onslide<1->{
%    } % onslide 2
%    \onslide<1->{
%    } % onslide 3
%    \onslide<1->{
%    } % onslide 4
%    \onslide<1->{
%    } % onslide 5
%    \onslide<1->{
%    } % onslide 6
%    \onslide<1->{
%    } % onslide 7
%    \onslide<1->{
%    } % onslide 8
%    \onslide<1->{
%    } % onslide 9
%    \onslide<1->{
%    } % onslide 10
% \end{frame}



% \begin{frame}
% % \frametitle{d}
%    \onslide<1->{
%    } % onslide 2
%    \onslide<1->{
%    } % onslide 3
%    \onslide<1->{
%    } % onslide 4
%    \onslide<1->{
%    } % onslide 5
%    \onslide<1->{
%    } % onslide 6
%    \onslide<1->{
%    } % onslide 7
%    \onslide<1->{
%    } % onslide 8
%    \onslide<1->{
%    } % onslide 9
%    \onslide<1->{
%    } % onslide 10
% \end{frame}


\end{document}
